\documentclass[a4paper,10pt]{report}
\usepackage[utf8]{inputenc}
\usepackage[T1]{fontenc}      % caractères français
\usepackage{geometry}         % marges

\usepackage{graphicx}         % images
\usepackage{verbatim}         % texte préformaté
% Title Page
\title{Visualisation de Graphe en 2D}
\author{Torko Hervé , Abak-kali Nizar}
\date{17/01/2014}  

\begin{document}
\maketitle
\tableofcontents
\chapter{Introduction}
  \section{Objectif du projet}
    \paragraph{On souhaite visualiser Un graphe en 2D}
     
      \begin{itemize}
	\item[-]Pour cela, on doit placer les noeuds de maniere optimal selon un certain nombre de critère:
	  \subitem[-] éviter les croisements d'arcs
 	  \subitem[-] les noeuds reliés entre eux doivent êtres proches 
	\item[-] On pourra utiliser un modèle physique : 
	   \subitem[-] elastique : les arcs se comporte comme des elastiques
	    \subitem[-] gravitation les noeuds ont des champs de gravitation qui attire ou repousse les autres noeuds selon si il y a liason ou non 
	    \subitem[-] les noeuds non liés se repoussent 
	 \item[-] Il faura determiner la qualités d``un graphe selon certain critère
	 \item[-] Une partie graphique permettra de visualiser l'evolution du graphe 
      \end{itemize}
  \section{Notre Projet}
    \subsection{Presentation du Projet}	
      \subsubsection{Outils Utilise}
	 \paragraph{Langage utilisé}
	  Lors de notre projet nous avons programmé en C++ ,pour deux raisons :
	    \begin{itemize}
	      \item[->] Le langage est plus facile de prise en main et contient beaucoup plus outils que le C .Deplus,
	      le C++ contient beaucoup de structure prefaite très utile (ex: list , vector , map,etc..)
	      \item[->] Pour ce projet nous allons utiliser un outils qui nous a été très utile pour travailler et 
	      manipuler des graphes: LEMON \footnote[1]{Library for Efficient Modeling and Optimization in Networks}.
	      Or LEMON est present uniquement en C++ .
	    \end{itemize}
	 \paragraph{Bibliothèques SDL}
	  Pour repondre au besoins d'une partie graphique nous avons opté pour SDL . Il est simple d'utilisation , nous savons l'utiliser,
	  et on disposait deja d'une petite bibliothèque de fonctions graphique.
\chapter{Explication du Code}
  \section{auxi.cpp}
    \paragraph{}
      Il s'agit du fichier qui contient les fonctions de dessin tel que circle pour des cercles , line pour dessiner des ligne, arrow pour des lignes flechées.
      Ces fonctions ont été trouvées sur le site de Mr Audibert .
    \subsection{Structure utilisé}
     Aucune structure particulière dans ce fichier n'est utilisé.
    \subsection{Fonctions utilisé}
     \subsubsection{putpixel()}
      \begin{scriptsize}
	\begin{verbatim}
	 void putpixel(SDL_Surface *ecran,int xe, int ye, Uint32 c) {
  Uint32 * numerocase;
  numerocase = (Uint32*)(ecran->pixels)+xe+ye*ecran->w;
  *numerocase = c;
}
	\end{verbatim}
	  \end{scriptsize}
      Il s'agit de la base des fonctions de dessin. Elle permet de poser une couleur à une position donné.\\
    \subsubsection{getpixel()}  
       \begin{scriptsize}
	\begin{verbatim}
	Uint32 getpixel(SDL_Surface* ecran, int xe , int ye){
  Uint32 * numerocase;
  numerocase= (Uint32 *)(ecran->pixels)+xe+ye*ecran->w;
  return (*numerocase);

}	
	\end{verbatim}
      \end{scriptsize}
      La encore il s'agit d'une fonction de base qui consiste a recuperer la couleur d'un pixel dont on donne la position et à la retourné.\\
      
     \subsubsection{line()}
      \begin{scriptsize}
	\begin{verbatim}
	void line( SDL_Surface *ecran,int x0,int y0, int x1,int y1, Uint32 c) {
  int dx,dy,x,y,residu,absdx,absdy,stepx,stepy,i;
  dx=x1-x0; dy=y1-y0; residu=0;
  x=x0;y=y0; putpixel(ecran,x,y,c);
  if (dx>0) 
    stepx=1;
  else stepx=-1; 
  if (dy>0) 
    stepy=1; 
  else stepy=-1;
  absdx=abs(dx);
  absdy=abs(dy);
  if (dx==0) 
    for(i=0;i<absdy;i++) { 
      y+=stepy;
      putpixel(ecran,x,y,c);
    }
  else if(dy==0) 
    for(i=0;i<absdx;i++){ 
      x+=stepx;
      putpixel(ecran,x,y,c); 
    }
  else if (absdx==absdy)
    for(i=0;i<absdx;i++) {
      x+=stepx; y+=stepy;
      putpixel(ecran,x,y,c);
    }
  else if (absdx>absdy)
    for(i=0;i<absdx;i++)
      { x+=stepx; 
	residu+=absdy;
	if(residu >= absdx) 
	  {residu -=absdx; y+=stepy;}
	putpixel(ecran,x,y,c);
      }
  else for(i=0;i<absdy;i++)
	 {
	   y+=stepy; residu +=absdx;
	   if (residu>=absdy) 
	     {residu -= absdy;x +=stepx;}
	   putpixel(ecran,x,y,c);
	 }
}
    \end{verbatim}
      \end{scriptsize}
	il s'agit de la fonction qui permet de dessiner une ligne entre deux couples de coordonnées .
     \subsubsection{arrow}
      \begin{scriptsize}
	\begin{verbatim}
	void arrow(SDL_Surface *screen,int x1, int y1, int x2, int y2, Uint32 c)
{
  int dx,dy;
  float xf1,yf1,xf2,yf2,d,dx1,dy1,ndx1,ndy1,ndx2,ndy2,angle=M_PI/6.;
  line(screen,x1,y1,x2,y2,c);
  dx=x2-x1; dy=y2-y1; d=sqrt(dx*dx+dy*dy);
  if (d!=0.)
    { dx1=6.*(float)dx/d; dy1=6.*(float)dy/d;
      ndx1=dx1*cos(angle)-dy1*sin(angle);
      ndy1=dx1*sin(angle)+dy1*cos(angle);
      xf1=0.3*x1+0.7*x2; yf1=0.3*y1+0.7*y2; xf2=xf1-ndx1; yf2=yf1-ndy1;
      line(screen,xf1,yf1,xf2,yf2,c);
      ndx2=dx1*cos(-angle)-dy1*sin(-angle);
      ndy2=dx1*sin(-angle)+dy1*cos(-angle);
      xf2=xf1-ndx2; yf2=yf1-ndy2; line(screen,xf1,yf1,xf2,yf2,c);
    }
  else
    {cercle(screen,x1+10,y1,10,c); line(screen,x1+20,y1,x1+23,y1-6,c);
      line(screen,x1+20,y1,x1+15,y1-5,c);
    }
}

	\end{verbatim}
      \end{scriptsize}
      Cette fonction dessine un vecteur d'un couple de coordonnées A\{xa,ya\} B\{xb,yb\}  A->B .
  \section{phisic.cpp}
    Ce fichier contient toutes les fonctions physique du programme tel que le rapprochement ou l'éloignement de noeuds . 
    \subsection{Structure utilisé}
      \paragraph{droite_t}
	 
	  Cette structure est censé représenter une droite . 
	 Or une droite a pour formule  $y = (a * x) + b$ .
	  Donc ,notre structure contient deux doubles , un double qui represente le coefficient directeur et un autre qui correspond a l'intersection avec l'axe des ordonnées.   \subsection{Fonctions}
	  \begin{scriptsize}%structure
	   \begin{verbatim}
	    struct droite_t{
   double coeff;
  double inter;
  };
typedef struct droite_t drt;
	   \end{verbatim}

	   
	  \end{scriptsize}
      \paragraph{}
	Cette structure sert entre autre lors des rapprochements ou des eloignements pour pouvoir garder les deux noeuds sur un même axe .
	
	
      \subsection{Explication des fonctions}
	\subsubsection{Les fonctions de calcul de distance}
	  \begin{itemize}
	   \item distancepointcentre() :  distance entre les coordonnées d'un noeud et le centre de la fenêtre  
	   \item distance2points(): distance entre deux noeuds 
	  \end{itemize}
	\subsubsection{La fonction d'initialisation de la droite}
	  
	  \begin{scriptsize}
	    \begin{verbatim}
	     
drt initdroite(ListDigraph::Node noeud1, ListDigraph::Node noeud2, ListDigraph::NodeMap<int>& xe ,ListDigraph::NodeMap<int>& ye){
  drt f;
  //initialisation du coefficient directeur
  f.coeff = (ye[noeud1]-ye[noeud2])/(xe[noeud1]-xe[noeud2]);
  
//cacul intersection avec l'axe des ordonnees
   f.inter=-(f.coeff*xe[noeud1])+ye[noeud1];
  return f;
}
	    \end{verbatim}
	\end{scriptsize}
	\paragraph{}
	  cette fonction prends comme argument deux noeuds ainsi que les deux nodemaps de noeuds contenant les coordonnées de tout les noeuds .
	  cette fonction initialise la droite a partir des coordonnées des deux noeuds.
	  
	\subsubsection{Fonctions Physique}
	  \paragraph{}
	  
	  
	  
  \section{visu.cpp}
    \subsection{Structure utilisé}
    \subsection{Fonctions}
   
\chapter{Fonctionnement}
    \section{Mais comment que sa marche}
    \section{Quelques screenshots}

\chapter{Conclusion}
  \section{Ce qui a été fait}
  \section{Probleme rencontré}
  \section{Amelioration possible}
  
\end{document}          
